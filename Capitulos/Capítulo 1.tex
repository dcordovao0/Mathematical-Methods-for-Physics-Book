
\section{Introducción}
En las anteriores clases de física, hemos usado la siguiente notación para las coordenadas cartesianas:
\begin{align*}
    \mathbf{V}&=(V_x,V_y,V_z), \\
   \bmR&=(x,y,z).
\end{align*}
Aunque esta notación sea la más natural e intuitiva, hay varias razones (que se expondrán más adelante) para cambiarla de la siguiente manera:
\begin{align*}
    \mathbf{V}&=(V_1,V_2,V_3), \\
    \bmR&=(x_1,x_2,x_3).
\end{align*}
\section{Producto Punto}
El producto punto en $\mathbb{R}^3$ viene dado por:
\begin{equation}
    \mathbf{a}\cdot\mathbf{b}=a_1b_1+a_2b_2+a_3b_3=\sum_{i=1}^3a_ib_i.
\end{equation}
Se puede notar que se puede simplificar la notación. Primero, podemos omitir el símbolo de sumatoria, ya que sabemos de antemano que estamos trabajando en el espacio tridimensional y los índices repetidos nos indican implícitamente que hay una suma. Con esta simplificación tenemos
\begin{equation}
    \mathbf{a}\cdot\mathbf{b}=a_ib_i
\end{equation}
al subíndice $i$ se lo conoce como índice mudo, ya que se lo puede cambiar por cualquier símbolo y la ecuación mantiene su significado. A esta simplificación se la conoce como notación de Einstein o notación de índices repetidos. Si queremos más generalidad en la notación, en lugar de repetir índices podemos usar la delta de Kronecker que viene definida por
\begin{equation}
    \delta_{ij}=
    \begin{cases}
        1, & \text{si }i=j, \\
        0, & \text{si }i\neq j,
    \end{cases}
\end{equation}
usando esta definición escribimos
\begin{equation}
    \mathbf{a}\cdot\mathbf{b}= \delta_{ij}a_ib_j.
\end{equation}
\section{Producto Cruz}
Ahora, nos interesa saber si es que esta notación puede ayudarnos a simplificar el producto cruz, porque no es de mucha ayuda si es que solo simplifica la escritura del producto punto. Normalmente, se calcula el producto cruz de la siguiente manera
\begin{equation}
     \mathbf{a}\times\mathbf{b}=
     \begin{vmatrix}
        \hat{x_1} & \hat{x_2} & \hat{x_3}\\
        a_1 & a_2 & a_3 \\
        b_1 & b_2 & b_3 
     \end{vmatrix}
     =\mathbf{c}=
     \begin{pmatrix}
         c_1 \\
         c_2 \\
         c_3
     \end{pmatrix}
     =\begin{pmatrix}
         a_2b_3-a_3b_2 \\
         a_3b_1-a_1b_3 \\
         a_1b_2-a_2b_1.
     \end{pmatrix}
\end{equation}
Queremos encontrar un objeto que satisfaga la igualdad
\begin{equation}
    c_i= ? a_jb_k,
\end{equation}
llamemos a este objeto $\varepsilon_{ijk}$ y veamos el caso de $c_1$ para definir las propiedades que debe tener el objeto para cumplir la igualdad:
\begin{equation}
    c_1=
    \begin{matrix}
        \varepsilon_{111} & \varepsilon_{112} & \varepsilon_{113} \\
        \varepsilon_{121} & \varepsilon_{122} & \varepsilon_{123} \\
        \varepsilon_{131} & \varepsilon_{132} & \varepsilon_{133} 
    \end{matrix}
    =\begin{matrix}
        a_1b_1 & a_1b_2 & a_1b_3 \\
        a_2b_1 & a_2b_2 & a_2b_3 \\
        a_3b_1 & a_3b_2 & a_3b_3.
    \end{matrix}
\end{equation}
Es claro que los únicos elementos que queremos que sean $\neq 0$ son $\varepsilon_{123}$ y $\varepsilon_{132}$ y que $\varepsilon_{123}$ debe ser positivo mientras que $\varepsilon_{132}$ debe ser negativo. Con estas consideraciones, definidos el símbolo de Levi-Civita de la siguiente manera:
\begin{equation}
    \varepsilon_{ijk}=
    \begin{cases}
        0, & \text{si }i=j \lor i=k \lor j=k , \\
        1, & \text{si }ijk \text{ es una permutación par de }123, \\
       -1, & \text{si }ijk \text{ es una permutación impar de }123.
    \end{cases}
\end{equation}
Definido este símbolo, podemos escribir el producto cruz cómo
\begin{equation}
    \mathbf{a}\times\mathbf{b}= \varepsilon_{ijk}a_jb_k,
\end{equation}
notamos que sólo hay 2 índices que se repiten, el tercer índice, en este caso $i$, es el que define que el resultado es un vector y no un escalar y recorre los componentes del vector. Este símbolo es muy útil para simplificar demostraciones de propiedades de operaciones vectoriales. Dos propiedades muy importantes de este símbolo son:
\begin{equation}
    \varepsilon_{ijk}=-\varepsilon_{ikj}
\end{equation}
y
\begin{equation}
     \varepsilon_{ijk} \varepsilon_{klm}=\delta_{il}\delta_{jm}-\delta_{im}\delta_{jl}.
\end{equation}
A continuación, unos ejemplos de la aplicación de los símbolos definidos. 

\begin{example}
    Demostrar que $\mathbf{a}\cdot(\mathbf{b}\times\mathbf{c})=(\mathbf{a}\times\mathbf{b})\cdot \mathbf{c}$.
\end{example} 

Pasamos todo a notación de Einstein y tenemos
\begin{align*}
    \mathbf{a}\cdot(\mathbf{b}\times\mathbf{c})&=\mathbf{a}\cdot(\varepsilon_{ijk}b_jc_k)\\
    &=a_i\varepsilon_{ijk}b_jc_k\\
    &=\varepsilon_{ijk}a_ib_jc_k\\
    &=(\varepsilon_{ijk}a_ib_j)c_k\\
    &=(\mathbf{a}\times\mathbf{b})c_k\\
    &=(\mathbf{a}\times\mathbf{b})\cdot \mathbf{c}.  
\end{align*}
Demostrando lo que se quería. \hfill $\blacksquare$

\begin{example}
    Demostrar que $\mathbf{a}\times(\mathbf{b}\times\mathbf{c})=\mathbf{b}(\mathbf{a}\cdot\mathbf{c})-\mathbf{c}(\mathbf{a}\cdot\mathbf{b})$.
\end{example} 

\begin{align*}
\mathbf{a}\times(\mathbf{b}\times\mathbf{c})&=\varepsilon_{ijk}a_j(\mathbf{b}\times\mathbf{c})_k\\
    &=\varepsilon_{ijk}a_j(\varepsilon_{klm}b_lc_m),
\end{align*}
usando la propiedad (1.11) y conmutando tenemos:
\begin{align*}
    (\varepsilon_{ijk}\varepsilon_{klm})a_jb_lc_m&=(\delta_{il}\delta_{jm}-\delta_{im}\delta_{jl})a_jb_lc_m\\
    &=\delta_{il}\delta_{jm}a_jb_lc_m-\delta_{im}\delta_{jl}a_jb_lc_m,
\end{align*}
por definción, $\delta_{jm}a_jc_m=a_jc_j$ y $\delta_{il}b_l=b_i$. Con esto se obtiene
\begin{align*}
    \delta_{il}\delta_{jm}a_jb_lc_m-\delta_{im}\delta_{jl}a_jb_lc_m&=b_i(a_jc_j)-c_i(a_lb_l)\\
    &=\mathbf{b}(\mathbf{a}\cdot\mathbf{c})-\mathbf{c}(\mathbf{a}\cdot\mathbf{b}). 
\end{align*}
Demostrando la propiedad. \hfill$\blacksquare$

La notación de Einstein también es útil para definir el operador nabla y sus operaciones asociadas. Definimos el operador nabla de la siguiente manera
\begin{equation}
    \nabla_j=\frac{\partial}{\partial x_j}.
\end{equation}
Con esto definimos el gradiente, la divergencia y el rotacional:
\begin{equation}
    \nabla_jf(\bmR)
\end{equation}
\begin{equation}
    \nabla_ia_i(\bmR)=\delta_{ij}\frac{\partial}{\partial x_i}a_j(\bmR)
\end{equation}
\begin{equation}
    \varepsilon_{ijk}\nabla_ja_k(\bmR)=\varepsilon_{ijk}\frac{\partial}{\partial x_k}a_k(\bmR)
\end{equation}